\part{Aspectos Gerais}

\chapter[Aspectos Gerais]{Aspectos Gerais}

Estas instruções apresentam um conjunto mínimo de exigências necessárias a 
uniformidade de apresentação do relatório de Trabalho de Conclusão de Curso 
da FGA. Estilo, concisão e clareza ficam inteiramente sob a 
responsabilidade do(s) aluno(s) autor(es) do relatório.

As disciplinas de Trabalho de Conclusão de Curso (TCC) 01 e Trabalho de 
Conclusão de Curso (TCC) 02 se desenvolvem de acordo com Regulamento 
próprio aprovado pelo Colegiado da FGA. Os alunos matriculados nessas 
disciplinas devem estar plenamente cientes de tal Regulamento. 

\section{Construção da Proposta}
\label{sec:constru_o_da_proposta}
A Rede Social About (RSA) tem o propósito de dar transparência às personalidades de seus usuários, permitindo com que todos
estes saibam sobre qualquer aspécto sobre qualquer outro usuário, dede que ambos tenham aceitado os
termos de consentimento pré estabelecidos.

Qualquer usuário pode criar abouts para seus amigos, que foram previamente aceitos. A fim de julgar
sobre a veracidade deste about, os demais amigos do usuário poderão votar sobre a veracidade,
julgando se é uma verdade ou uma mentira. Tanto o about escrito quando o voto são anônimos para todos os
usuários.

A proposta deste trabalho é desenvolver e aplicar uma instância do framework octalysis de gamificação na RSA, engajando e
motivando seus usuários a executarem determinadas tarefas que, serão discutidas posteriormente.

Os seguintes
passos serão estabelecidos para a execução proposta:

\begin{enumerate}
    \item Execução do teste piloto
    \item Survey para identificação das técnicas de gamificação
    \item Análise estatística das técnicas
    \item Construção do framework de gamificação
    \item Escolha do objeto de gamificação
    \item Implementação das técnicas no objeto de gamificação
\end{enumerate}

\subsection{Execução do Piloto}
\label{sub:execu_o_do_piloto}
A ideia sobre a rede social surgiu de uma simples ideia de um aluno de graduação. Desta forma, se até mesmo a
ideia era impalpável, algumas perguntas surgem sobre a gamificação desta, por exemple:

\begin{itemize}
    \item Como conseguir averiguar como traços de gamificação para ser aplicado?
    \item Como identificar quais são as motivações básicas que devem ser levadas em consideração?
    \item Para atingir tais motivações, quais são as técnicas que devem ser adotas?
\end{itemize}

Para responder essas perguntas, que irão servir de insumo para o direcionamento dos pilares da gamificação, será elaborado um projeto pilotor, tal qual terá algumas funcionalidades básicas da RSA, sem levar em consideração requisitos não funcionais como: segurança, usabilidade, performance e padrões de design.

O objetivo é proporcionar aos usuários um cenário similar, quanto a funcionalidade, ao que estes irão utilizar
na RSA, a fim de identificar quais as técnicas de gamificação são mais presentes.

Para desenvolver e aplicar este piloto, será necessário executar um procedimento com os seguintes passos:
% definir tecnologia,
% desenvolver a solução, virá-la para produção, aplicar marketing da solução para um público reduzido, manter solução, finalizar solução.

\begin{itemize}
    \item Definir tecnologia
    \item Desenvolver solução
    \item Virá-la para produção
    \item Aplicar marketing da solução para um público reduzido
    \item Manter solução
    \item Finalizar solução
\end{itemize}

Estes pontos necessários para que a implementação da solução do projeto piloto serão detalhados a seguir:

\subsubsection{Definir Tecnologia}
\label{sub:definir_tecnologia}
Como o desenvolvimento da solução para o projeto piloto não necessita conter interfaces com design bem elaborado;
usabilidade, baseado em user experience, elavada; dentre outras necessidades não funcionais. Os critérios levados em
consideração para a sua construção foram os seguintes:

\begin{itemize}
    \item Desenvolvimento rápido: por se tratar de uma simples solução, temporária, o tempo de implementação deveria
        ser curto, levantando o requisito de utilizar um framework de alta produtividade.
    \item Padrões de Design simples: por se tratar de uma aplicação extremamente funcional, tal que o usuário não irá
        utilizá-la por muito tempo, não há a necessidade de fortes esforços e grande elaboração dos padrões de
        User Experience.
    \item Escalabilidade baixa: como se trata de um público pequeno, para poucos usuários simultâneos, em torno de duzentos
        acessos diários, o framework pode ser de baixo desenpenho, facilitanto assim a sua escolha.
    \item Os usuários não carecem de executar cadastros e logins: para o escopo do piloto, cada usuário não necessitará
        fazer registros e logins no site. A política de gerenciamento dos votos será feita mediante armazenamento dos
        IP's utilizados durante o acesso, fazendo com que cada IP possa votar apenas uma vez dentro de um determinado
        intervalo de tempo.
    \item Suporte para questionários: como o piloto se trata de quistionários baseados em perguntas, há a necessidade que o framework
        escolhido conceda suporte para criação de perguntas, possibilidade de votação, vizualização e contagem dos votos
        separados por perguntas.
    \item Facilidade de implantação: como se trata de um site web bem rápido, há a necessidade de que este seja de fácil e rápida
        implantação, possibilitando que rápidamente seja colocado em produção. Dessa forma, o framework escohido também carece
        de propiciar suporte para esta feature.
\end{itemize}

Dados os requisitos acima, irá se iniciar um processo de busca por ferramentas e frameworks que possibilitem a implentação do site
web que os contemple. Para a avaliação, as ferramentas levantadas, que serão até cinco, irão ser submetidas à uma tabela que contempla
os requisitos básicos necessários. Cada item da tabela poderá ser julgado de um a cinco, sendo que a nota 1 diz que a ferramenta não nada
desta funcionalidade, já a nota 5 diz que esta contém totalemente esta funcionalidade. Assim, os valores intermediários: 2, 3 e 4 representam
que possuem o requisito parcialmente, de acordo com a nota.

% \usepackage{booktabs}
\begin{table}[]
    \centering
    \begin{tabular}{@{}llllll@{}}
        \toprule
        \textbf{}                                                  & \textbf{Tool A}       & \textbf{Tool B}       & \textbf{Tool C}       & \textbf{Tool D}       & \textbf{Tool E}       \\ \midrule
        \multicolumn{1}{|l|}{\textbf{Desenvolvimento Rápido}}      & \multicolumn{1}{l|}{} & \multicolumn{1}{l|}{} & \multicolumn{1}{l|}{} & \multicolumn{1}{l|}{} & \multicolumn{1}{l|}{} \\ \midrule
        \multicolumn{1}{|l|}{\textbf{Padrões de Design Simpels}}   & \multicolumn{1}{l|}{} & \multicolumn{1}{l|}{} & \multicolumn{1}{l|}{} & \multicolumn{1}{l|}{} & \multicolumn{1}{l|}{} \\ \midrule
        \multicolumn{1}{|l|}{\textbf{Sem Autenticação}}            & \multicolumn{1}{l|}{} & \multicolumn{1}{l|}{} & \multicolumn{1}{l|}{} & \multicolumn{1}{l|}{} & \multicolumn{1}{l|}{} \\ \midrule
        \multicolumn{1}{|l|}{\textbf{Baixa Escalabilidade}}        & \multicolumn{1}{l|}{} & \multicolumn{1}{l|}{} & \multicolumn{1}{l|}{} & \multicolumn{1}{l|}{} & \multicolumn{1}{l|}{} \\ \midrule
        \multicolumn{1}{|l|}{\textbf{Suporte para Questionário}}   & \multicolumn{1}{l|}{} & \multicolumn{1}{l|}{} & \multicolumn{1}{l|}{} & \multicolumn{1}{l|}{} & \multicolumn{1}{l|}{} \\ \midrule
        \multicolumn{1}{|l|}{\textbf{Facilidade de Implementação}} & \multicolumn{1}{l|}{} & \multicolumn{1}{l|}{} & \multicolumn{1}{l|}{} & \multicolumn{1}{l|}{} & \multicolumn{1}{l|}{} \\ \midrule
                                                                   &                       &                       &                       &                       &                       \\ \midrule
                                                                   \multicolumn{1}{|l|}{\textbf{Total}}                       & \multicolumn{1}{l|}{} & \multicolumn{1}{l|}{} & \multicolumn{1}{l|}{} & \multicolumn{1}{l|}{} & \multicolumn{1}{l|}{} \\ \bottomrule
    \end{tabular}
    \caption{Avaliação dos Frameworks}
    \label{my-label}
\end{table}

Cada ferramenta, no final da avaliação, terá uma nota entre 6 e 30, que será disposta na linha 'Total'. Assim, a ferramenta que possuir a maior nota será escolhida para a execução do projeto piloto.

\subsubsection{Desenvolver Solução}
\label{sub:definir_tecnologia}

Dada a ferramenta escolhida, faz-se necessário que esta seja analizada em termos técnicos. Será necessário analisar e seguir alguns pontos, que
serão descritos abaixo para que a solução seja implementada com sucesso:

\begin{enumerate}
    \item Escolher versão do framework que será utilizado;
    \item Executar download do framework para o laboratório local, que terá os testes executados;
    \item Executar instalação da ferramenta em um laboratório local, que será utilizado como ambiente de
        desenvolvimento da aplicação;
    \item Definir qual template será utilizado para a página home do site, layout da aplicação e menu principal;
    \item Executar o download do plugin de execução de questionários na página principal do framework que será escolhido;
    \item Instalar na aplicação o plugin para a criação, manutenção e vizualização dos questionários;
    \item Configurar plugin de questionários para armazenar as perguntas e os índices de votação de cada pergunta em persistência;
        Para que posteriormente seja possível executar a análise de todos os dados coletados;
    \item Executar a criação de um questionário a fim de homologar a solução desenvolvida para os pré-requisitos estabelecidos. 
    \item Executar a integração da questão criada para homologação no layout da home da aplicação;
    \item Executar o gerenciamento de configuração de software para que o código fonte seja armazenado. Este será posteriormente
        capturado para executar a aplicação no servidor de produção;
\end{enumerate}

Com todos esses passos executados, a solução está operando de acordo como o esperado para recolher os dados básicos propostos anteriormente.
Assim, esta está preparada para ser disposta em produção em um servidor e disponibilizá-la para o público geral. Na próxima sessão serão
detalhados os passos para ter a aplicação em produção.

\subsubsection{Virá-la para a Produção}
\label{sub:definir_tecnologia}
Para que qualquer pessoa com acesso à internet possa conseguir ter acesso à aplicação desenvolvida, faz-se necessário que
o site hospedado esteja em um servidor com um IP externo válido. Para melhor utilização da plataforma, será necessário adquirir um domínio
que faça o apontamento para o IP do servidor adquirido. 

Os passos necessários para virar o servidor de produção serão descritos nos itens seguintes:

\begin{enumerate}
    \item Avaliar qual será o provedor de máquinas virtuais será utilizado;
    \item Adquirir uma máquina virtual com IP externo. Este deve conter o mínimo possível de capacidade de processamento e
        disponibilidade de memória RAM para que seja possível suportar a hospedagem da solução;
    \item Adquirir um domínio em um servidor DNS do server .com.br para apontamento do IP externo;
    \item Configurar o domínio DNS adquirido para que este execute o apontamento do IP do servidor que será utilizado;
    \item Executar instalação de um servidor de páginas HTTP no servidor;
    \item Executar a instalação de uma base de dados para armazenamento das informações obtidas;
    \item Recuperar o código fonte utilizado no ambiente de desenvolvimento para o servidor. Este será devidamente
        instalado e recuperado assim como foi feito anteriormente;
    \item Executar as configurações do framework para que este opere corretamente utilizando um servidor externo;
    \item Executar a criação novamente de uma questão para que seja possível homologar o ambiente de produção.
\end{enumerate}

Estes passos vão assegurar que o servidor seja configurado corretamente e que esteja disponível para acesso externo para
todos os usuários que vão o sistema.

Com os procedimentos necessários para que o projeto piloto esteja acessível pelos usuários, já será possível executar o marketing
para divulgar a algumas pessoas a aplicação. Os passos para o marketing serão descritos no próximo sub tópico.

\subsubsection{Aplicar Marketin do projeto piloto}
\label{sub:definir_tecnologia}
Como se faz necessário que haja usuários utilizando o protótipo para recolher os dados, é fundamental que o propósito
e o protótipo sejam divulgados para o público externo que irá utilizá-lo. 

A proposta de marketing seguirá algumas diretivas que serão apresentadas a seguir:

\begin{enumerate}
    \item O público alvo foi definido para que este pudesse ser atingido facilmente. Como estamos tratando de um
        projeto de desenvolvendo universitário, este meio pode ser facilmente almejado em pouco tempo. Isto se
        deve ao volume de alunos existentes no campus com disponibilidade para testar novos projetos e ideias.
        Desta maneira, o público alvo serão os universitários da UnB unidade Gama - Distrito federal(UnB-FGA).
    \item Como estamos tratando de um projeto piloto que carece da presença de usuários a utilizando, alguns
        pontos são extremamente importante para que o público alvo definido seja atingido. Assim, o meio de
        distribuição do protótipo deve conter os seguintes pontos:
        \begin{itemize}
            \item Ser possível compartilhar os links publicados no protótipo;
            \item Novas enquentes devem chegar rapidamente ao público alvo;
            \item Novas enquentes devem ser dispostas em um meio que esteja disponível para todos os
                alunos do campus FGA.
        \end{itemize}
    \item Existem vários meios possíveis para alcançar o público alvo, por exemplo:
        \begin{itemize}
            \item Contato direto verbal;
            \item Cartazes e planfetos no campus;
            \item Listas de emails;
            \item Sites de fóruns da UnB;
            \item Sites de devulgação;
            \item Grupos e páginas do facebook.
        \end{itemize}
    \item Desta forma, será escolhido o meio de comunicação que atenda de melhor maneira os pré-requisitos descritos acima. Para que
        seja possível compartilhar links, demonstrar rapidamente novas enquente e com o maior número de estudantes, o melhor meio de
        comunicação é a utilização do facebook. Utilizando o facebook, é possível utilizar o grupo da faculdade, que contém vários alunos,
        levantando em consideração que os links podem ser compartilhados por lá. Dessa forma, toda a apresentação do protótipo será executada
        via grupo da faculdade no facebook;
    \item Todas as novas enquetes serão apresentadas no grupo e compartilhadas também no grupo do facebook do campus da faculdade.
\end{enumerate}

Assim, todas as novas enquetes irão seguir os padrões estabelecidos nestas diretrízes de regras de marketing. Isto irá assegurar que
o público alvo escolhido seja aumejado.
 

\subsubsection{Manter a Solução}
\label{sub:definir_tecnologia}
Após a construção estabelecida e disponível para que os usuários a utilizem, já é possível aplicar novos questionários, fazendo uso do plano de marketing.
Esta etapa consistirá em criar um sistema em que os próprios usuários vão ceder as novas informações para novos questionários. Estes questionários 
terão as informações coletadas e futuramente utilizadas.

Primeiramente, será criada uma segunda página para o site. Esta página será responsável por conter uma enquente com a seguinte pergunta:

 \begin{quote}
     "Qual deve ser a próxima enquete do site?"
 \end{quote}

As perguntas dispostas serão analisadas e as que foram consideradas de bom gosto, serão utilizadas. A cada dia, uma nova questão será aprensentada
para a enquente. Essa nova enquente será publicada e compartilhada. Após 48 horas, esta será dada como finalizada e o resutaldo será apresentado
para o público. 

Este ciclo será mantido por duas semanas, possibilitando com que sejam elaborados materias suficientes para recolher os indicadores que necessitamos 
das técnicas de gamificação da rede social.
\subsubsection{Finalizar a Solução}
\label{sub:definir_tecnologia}

Após duas semanas de uso da solução proposta, os dados serão recolhidos e o servidor será desligado para evitar gastos. O domínio continuará em operação
por mais um ano, porém, não apontará para um endereço de IP válido, pois, o servidor que conterá o endereço externo não estará mais em operação.

Este será o tempo necessário para implantar e recolher todas as informações propostas para o uso da solução de projeto piloto.

\subsection{Levantamento das Técnicas de Gamificação}
\label{sub:survey_para_t_cnicas_de_gamifica_o}
Será executado um levantamento com alguns usuários aleatoriamente escolhidos dentre os que interagiram com o piloto.
O objetivo do levantamento é identificar quais são as técnicas de gamificação mais presentes no piloto executado.

O levantamento será executado em duas etapas. A primeira se trata de conseguir entender o que os usuários entendem e pensam sobre o objetivo
principal que o projeto piloto terá a intenção de retratar. A segunda parte consistirá na elaboração de um survey com opções de valores entre 1 e 5,
listando todas as técnicas de gamificação existentes no octalysis.

Os procedimentos sobre como serão elaboradas as duas próximas etapas serão descritas nas duas sessões seguintes. 

\subsubsection{Características Projeto do Piloto}
\label{sub:caracter_sticas_projeto_do_piloto}
Com a intenção de compreender a visão que os usuários irão ter do projeto, bem como entender onde podem ser trabalhadas suas motivações básicas,
serão levantadas as suas características.

Este processo será elaborado fazendo com que os usuários respondam três perguntas abertas. As perguntas são as seguintes:

\begin{quotation}
    Questão 01: Na sua opinião, o que este site representava?
\end{quotation}

\begin{quotation}
    Questão 02: O que você acredita que motivava e levava as pessoas a utilizarem o site?
\end{quotation}

\begin{quotation}
    Questão 03: E quanto ao contrário, o que você acredita que levava as pessoas a se desmotivarem e a não
    utilizar mais o site?
\end{quotation}

Estas questões serão feitas a alguns usuários do sistema individualmente. As suas respostas serão gravadas para futuras análises.
Além disso, as respostas serão transcrevidas para o relatório.

\subsection{Survey das Técnicas}
\label{sub:survey_das_t_cnicas}



\subsection{Análise Estatística das Técnicas}
\label{sub:an_lise_estat_stica_das_t_cnicas}
A partir da massa de dados obtida com o survey, serão executadas análises estatícas a fim de identificar correlações
entre as regras e permitir que seja possível entender quais são as técnicas mais presentes.

\subsection{Construção do Framework}
\label{sub:constru_o_do_framework}
Utilizando as análises estatísticas realizadas com base no survey, serão extraídos os dados de quais técnicas de gamificação
devem ser mais presentes na RSA.

O objetivo é  utilizar técnicas que permitam que haja uma forte correlação entre si.

\subsection{Objeto de Gamificação}
\label{sub:objeto_de_gamifica_o}
Dadas as técnicas escolhidas, será analisado pelo proprietário do produto qual é o melhor objeto a ser gamificado na RSA.

Este objeto será alvo das técnicas e das implementações para desenvolver as motivações básicas necessárias.

\subsection{Implementação das Técnicas}
\label{sub:implementa_o_das_t_cnicas}
A partir do objeto escolhido, será possível implementar o código que fará a RSA ser gamificada.







\section{Composição e estrutura do trabalho}

A formatação do trabalho como um todo considera três elementos principais: 
(1) pré-textuais, (2) textuais e (3) pós-textuais. Cada um destes, pode se 
subdividir em outros elementos formando a estrutura global do trabalho, 
conforme abaixo (as entradas itálico são \textit{opcionais}; em itálico e
negrito são \textbf{\textit{essenciais}}):

\begin{description}
	\item [Pré-textuais] \

	\begin{itemize}
		\item Capa
		\item Folha de rosto
		\item \textit{Dedicatória}
		\item \textit{Agradecimentos}
		\item \textit{Epígrafe}
		\item Resumo
		\item Abstract
		\item Lista de figuras
		\item Lista de tabelas
		\item Lista de símbolos e
		\item Sumário
	\end{itemize}

	\item [Textuais] \

	\begin{itemize}
		\item \textbf{\textit{Introdução}}
		\item \textbf{\textit{Desenvolvimento}}
		\item \textbf{\textit{Conclusões}}
	\end{itemize}

	\item [Pós-Textuais] \
	
	\begin{itemize}
		\item Referências bibliográficas
		\item \textit{Bibliografia}
		\item Anexos
		\item Contracapa
	\end{itemize}
\end{description}

Os aspectos específicos da formatação de cada uma dessas três partes 
principais do relatório são tratados nos capítulos e seções seguintes.

No modelo \LaTeX, os arquivos correspondentes a estas estruturas que devem
ser editados manualmente estão na pasta \textbf{editáveis}. Os arquivos
da pasta \textbf{fixos} tratam os elementos que não necessitam de 
edição direta, e devem ser deixados como estão na grande maioria dos casos.

\section{Considerações sobre formatação básica do relatório}

A seguir são apresentadas as orientações básicas sobre a formatação do
documento. O modelo \LaTeX\ já configura todas estas opções corretamente,
de modo que para os usuários deste modelo o texto a seguir é meramente
informativo.

\subsection{Tipo de papel, fonte e margens}

Papel - Na confecção do relatório deverá ser empregado papel branco no 
formato padrão A4 (21 cm x 29,7cm), com 75 a 90 g/m2.

Fonte – Deve-se utilizar as fontes Arial ou Times New Roman no tamanho 12 
pra corpo do texto, com variações para tamanho 10 permitidas para a 
wpaginação, legendas e notas de rodapé. Em citações diretas de mais de três 
linhas utilizar a fonte tamanho 10, sem itálicos, negritos ou aspas. Os 
tipos itálicos são usados para nomes científicos e expressões estrangeiras, 
exceto expressões latinas.

Margens - As margens delimitando a região na qual todo o texto deverá estar 
contido serão as seguintes: 

\begin{itemize}
	\item Esquerda: 03 cm;
	\item Direita	: 02 cm;
	\item Superior: 03 cm;
	\item Inferior: 02 cm. 
\end{itemize}

\subsection{Numeração de Páginas}

A contagem sequencial para a numeração de páginas começa a partir da 
primeira folha do trabalho que é a Folha de Rosto, contudo a numeração em 
si só deve ser iniciada a partir da primeira folha dos elementos textuais. 
Assim, as páginas dos elementos pré-textuais contam, mas não são numeradas 
e os números de página aparecem a partir da primeira folha dos elementos 
textuais que é a Introdução. 

Os números devem estar em algarismos arábicos (fonte Times ou Arial 10) no 
canto superior direito da folha, a 02 cm da borda superior, sem traços, 
pontos ou parênteses. 

A paginação de Apêndices e Anexos deve ser contínua, dando seguimento ao 
texto principal.

\subsection{Espaços e alinhamento}

Para a monografia de TCC 01 e 02 o espaço entrelinhas do corpo do texto 
deve ser de 1,5 cm, exceto RESUMO, CITAÇÔES de mais de três linhas, NOTAS 
de rodapé, LEGENDAS e REFERÊNCIAS que devem possuir espaçamento simples. 
Ainda, ao se iniciar a primeira linha de cada novo parágrafo se deve 
tabular a distância de 1,25 cm da margem esquerda.

Quanto aos títulos das seções primárias da monografia, estes devem começar 
na parte superior da folha e separados do texto que o sucede, por um espaço 
de 1,5 cm entrelinhas, assim como os títulos das seções secundárias, 
terciárias. 

A formatação de alinhamento deve ser justificado, de modo que o texto fique 
alinhado uniformemente ao longo das margens esquerda e direita, exceto para 
CITAÇÕES de mais de três linhas que devem ser alinhadas a 04 cm da margem 
esquerda e REFERÊNCIAS que são alinhadas somente à margem esquerda do texto 
diferenciando cada referência.

\subsection{Quebra de Capítulos e Aproveitamento de Páginas}

Cada seção ou capítulo deverá começar numa nova pagina (recomenda-se que 
para texto muito longos o autor divida seu documento em mais de um arquivo 
eletrônico). 

Caso a última pagina de um capitulo tenha apenas um número reduzido de 
linhas (digamos 2 ou 3), verificar a possibilidade de modificar o texto 
(sem prejuízo do conteúdo e obedecendo as normas aqui colocadas) para 
evitar a ocorrência de uma página pouco aproveitada.

Ainda com respeito ao preenchimento das páginas, este deve ser otimizado, 
evitando-se espaços vazios desnecessários. 

Caso as dimensões de uma figura ou tabela impeçam que a mesma seja 
posicionada ao final de uma página, o deslocamento para a página seguinte 
não deve acarretar um vazio na pagina anterior. Para evitar tal ocorrência, 
deve-se re-posicionar os blocos de texto para o preenchimento de vazios. 

Tabelas e figuras devem, sempre que possível, utilizar o espaço disponível 
da página evitando-se a \lq\lq quebra\rq\rq\ da figura ou tabela. 

\section{Cópias}

Nas versões do relatório para revisão da Banca Examinadora em TCC1 e TCC2, 
o aluno deve apresentar na Secretaria da FGA, uma cópia para cada membro da 
Banca Examinadora.

Após a aprovação em TCC2, o aluno deverá obrigatoriamente apresentar a 
versão final de seu trabalho à Secretaria da FGA na seguinte forma:

\begin{description}
	\item 01 cópia encadernada para arquivo na FGA;
	\item 01 cópia não encadernada (folhas avulsas) para arquivo na FGA;
	\item 01 cópia em CD de todos os arquivos empregados no trabalho;
\end{description}

A cópia em CD deve conter, além do texto, todos os arquivos dos quais se 
originaram os gráficos (excel, etc.) e figuras (jpg, bmp, gif, etc.) 
contidos no trabalho. Caso o trabalho tenha gerado códigos fontes e 
arquivos para aplicações especificas (programas em Fortran, C, Matlab, 
etc.) estes deverão também ser gravados em CD. 

O autor deverá certificar a não ocorrência de “vírus” no CD entregue a 
secretaria. 

