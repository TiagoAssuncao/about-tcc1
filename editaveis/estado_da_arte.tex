\chapter[Estado da Arte]{Estado da Arte}
Esta sessão trará todo referêncial teórico necessário para desenvolver o projeto 
de medição de débito técnico, contemplando tópicos explicativos sobre Débito Técnico,
Contratação e Medição e Análise.

\section{Fases da Gamificação}
\label{sub:fasesgamification}
Todos os produtos que as pessoas utilizam na internet possuem diferentes
fases ao longo do seu ciclo de vida. Cada fase é reponsável por um tipo de contato diferente 
do usuário com a interface e com a imersão em que este está submetido.

Cada fase representa um sentimento diferente, uma experiência diferente
e uma nova forma de se lidar com aqueles atributos referentes ao que está
sendo lidado no procedimento de interação com o produto propriciado.

Essas fases que cada projeto é submetido já são conhecidos e desenhados. As fases
são quatro, bem claras e definidas. Elas são as seguintes:

\begin{enumerate}
    \item Descoberta;
    \item Reconhecimento;
    \item Construção;
    \item Fim de jogo.
\end{enumerate}

Essas fases circundam o ciclo de vida de um produto, desde o momento que este
é apresentado ao público até o momento que é deixado por ele. 

A definição das fases será ilustrada claramente nos subcapítulos que virão seguir.

\subsection{Descoberta}
\label{sub:descoperta}
É a fase onde o usuário não conhece sobre o produto, naõ tem noção de quais são 
os
seus objetivos nem como pode utilizá-lo. Esta é a fase onde o usuário tem o primeiro
contato, onde percebe como este funciona, bom como seus conceitos e valores.

Um exemplo de descoberta é uma apresentação de uma página no facebook, onde,
o novo produto será demostrado para grupos e nichos de interesse. A partir
de então, o usuário poderá passar a conhecer e utilizar o sistema.

Resumidamente, esta fase é reponsável por aprensentar o produto, fazer
com que os usuários o conheça.

\subsection{Reconhecimento}
\label{sub:reconhecimento}
Esta fase é reponsável por demonstrar ao usuário como o sistema se comporta.

Ela é essencial para que este entenda como o sistema funciona e o que cada
componente executa. Um exemplo bem conhecido desse procedimento é a utilização
de tutoriais e guias para novos usuários, no momento da sua chegada.

Ela termina quando o usuário está apto a continuar a utilizar o site sem
necessidade de aprender muitas outras novas ferramentas e funcionalidades.

Quando este está apto para tal, inicia-se a maior fase, onde o usuário
vai de fato entender e conhecer sobre o procedimento que está lidando.

\subsection{Construção}
\label{sub:constru_o}
Esta é a fase responsável pela real utilização do produto, onde as features
de fato serão utilizadas e irão agregar valor ao usuário.

Nesta parte o usuário já sabe e entende o papel de cada funcionalidade. Ele é capaz
de atingir os objetivos propostos. Aqui os recursos propostos serão utilizados
a depender na experiência e conexão do usuário com o produto.

Aqui tem que ser criados gatilhos para que mantenha o usuário constantemente utilizando
o sitema de acordo com o planejado.

\subsection{Fim de Jogo}
\label{sub:fim_de_jogo}
Toda aplicação desenvolvida passa pela fase de partida, onde está é totalmente utilizada
e de alguma forma, o usuário a deixará. 

Não necessariamente deixará de utilizar e participar do envovilmento total proposto pela
organização. Um exemplo disto é um jogo desenvolvido. Quando o primeiro jogo acabar, o
usuário passará pela fase de fim de jogo, que pode deixar o usuário motivado a se conectar
e adquirir a próxima versão do jogo que será lançada futuramente.

É importante que seja feito corretamente o desfeixo do produto para que uma linhagem seja
prosseguida.


Todas essas diretivas e fases que existem dentro do ciclo de vida de um produto 
deve ser
tratadas de forma independente e diferente entre si. Agora pode-se indagar onde 
a gamificação
entre neste processo, sendo que cada fase deve ser tratada de uma forma diferente pelo
usuário e, consecutivamente, por parte de quem está a oferecer o produto.

Assim, há a necessidade de que a gamificação também seja moldada conforme o objetivo de
cada fase a ser aplicada.

Dessa forma, cada fase implementada será pensada e avaliada para que seja possível 
aplicação de um  projeto de gamificação. Cada fase terá um foco em motivações
básicas diferentes, que propiciarão uma experiência diferente para o usuário.

A figura \ref{fig:fasesoctalysis} ilustra um exemplo do como pode ser aplicado na
Rede Social About a gamificação ao longo das quatro fases.

\begin{figure}[h]
    \centering
    \includegraphics[width=400px, scale=1]{figuras/fasesoctalysis}
    \caption{Fases do Octalysis}
    \label{fig:fasesoctalysis}
\end{figure}

Como pode ser visto na figura \ref{fig:fasesoctalysis}, são projetados vários
desenhos e desings modificados e diferentes para cada fase. Cada uma destas
tem um pensamento e objetivo diferente.

Na fase de descoberta, pode ser visto que a motivação básica mais presente é
a imprevisão e a curiosidade. O que dá margem para que o usuário imagine diferentes
possibilidades sobre o produto. 

No momento de uma propaganda, por exemplo, este lado do framework pode gerar uma

extrema curiosidade no usuário, o que fará com que ele fique motivado a procurar

e entender mais sobre o que está sendo anunciado.

Isto pode ser extremamente importante para conseguir capturar novos usuários.

Na segunda fase, em que o usuário vai conhecer sobre o produto, pode ser visto
que as fases relativas a desenvolvimento próprio e realização de si mesmo
são bem mais presentes.

Este ponto pode ser aplicado, pois o usuário irá se sentir realizado e inteligente
ao oberservar seu desenvolvimento próprio elevado. Isto irá gerar um prazer em fazê-lo
sentir o quanto pode ser bom em realizar as tarefas que a ele estão sendo design
adas
no início do procedimento.

Na terceira fase é possível verificar que duas motivações básicas são muito presentes:

\begin{itemize}
    \item Motivação Básica Cinco: Influência e Dinâmica Social;
    \item Motivação Básica Seis: Escassez e Impaciência.
\end{itemize}


Para a Motivação Básica Cinco, isto deixa o usuário motivado ao utilizar o produto
por sentir que está exercendo uma alta influência social, que está envolvido em
uma dinâmica social que faz influência em outras pessoas.

Isto faz com que o usuário fique motivado a continuar engajado no processo, pois
,
este estará conseguindo perceber o quanto está sendo participativo no meio social
e que o produto está sendo proveitoso por fazê-lo se sentir socialmente influente
e participativo.


A segunda motivação básica vizualizada nesta fase, Escassez e Impaciência, acontece
pois é possível verificar que o usuário fique motivado a executar determinadas
tarefas baseado neste sentimento.

Esta o deixará preocupado com a questão de não cumprir corretamente os objetivos
.
Esta fase é responsável por fazê-lo se sentir em um meio escasso caso não execute
os objetivos propostos.

Isto vai motivar o usuário e vai fazer com que faça o necessário para que não
sinta estes sentimentos.

A última fase, fim de jogo, também tem sua motivação básica predominante que
a guia. Esta é guiada pela Motivação Básica Oito: Perca e evitação.

Esta irá gerar um sentimento que faz o usuário se sentir mal. Este sentimento
envolve o fato de que o usuário pode perder o todo o processo que foi executado.



Este é um sentimento ruim. Sentimento qual o usuário não deseja sentir. Para tanto
ele se esforçará a fim de não presenciar as experiências que são submetidas.

Como pode ser visto, estes procedimentos de cada fase são extremamente aplicáveis
e úteis para que o usuário tenha várias experiências ao longo do clico de vida do
produto. O que propiciará uma experiência muito mais agradável.

Dessa forma, serão desenhados quatro frameworks diferentes para a Rede Social About.
Uma para cada fase diferente do produto, onde serão estudadas separadamente para

aplicá-las e possibilitar uma boa experiência para o usuário.

\section{Octalysis Strategy Dashboard}
\label{sec:octalysisdashborad}
O framework octalysis oferece suporte para a construção de um projeto de gamificação
bem estruturado e baseado em necessidades do domínio do problema.

Este suporte se trata do Octalysis Strategy Dashboard, o qual pode ser analisado
 as
estretégias de mergado, perspectiva do usuário, intenções desejadas para a gamificação,
mecanismos de feedback e incentivos.

Existem processos sistematizados para estabelecer cada fase e como será dado o 
resultado da gamificação. 

Para este trabalho, serão utilizados estes procedimentos sistematizados. 

Para ilustrar a metodologia de estratégia do octalysis dashboard, será representada
a figura a seguir, que contém a metodologia e a formalização da sua construção.

A seguir serão descritos sub capítulos, que retratarão o papel e a utilidade de 
cada
componente.


 \begin{figure}[h]
     \centering

     \includegraphics[width=450px, scale=1]{figuras/dashboard}
     \caption{Octalysis Strategy Dashboard}

     \label{fig:dashboard}
 \end{figure}

\subsection{Business Metrics}
\label{sub:business_metrics}
As métricas de negócio, em português, são termos quantitativos que podem ser utilizados
para ter um número palpaável sobre como está um determinado ponto do projeto de gamificação
que teve como o objetivo de ser atacado.

Essas métricas, irão auxiliar a verificarmos o quanto a aplicação da gamificação
 foi eficaz ou
não dentro de um determinado objetivo.

Alguns exêmplos de técnica de gamificação que serão utilizadas estão a seguir:

\begin{itemize}
    \item Aumentar o número de seguidores dos usuários prêmio
    \item Aumentar o número de vendas de um livro sobre o produto
    \item Aumentar o número de inscritos na rede social
    \item Aumentar a quantidade de acessos diários
    \item Aumentar os seguidores escritos
    \item Aumentar os usuários que compartilham conteúdos pelas redes sociais
    \item Aumentar a quantidade de curtidas em determinado post
\end{itemize}

Estes exemplos de métricas serão submetidos à Rede Social About antes da apresentação da
gamificação. E assim que determinada técnica for utilizada, será então executada
 uma
segunda medição, que propiciará analisar as diferenças entre os resultados obtidos.

\subsection{Define User Types}
\label{sub:define_user_types}
Este ponto do dashboard, para definir os tipos dos usuários, é responsável por conseguir
elaborar e definir quais são os tipos de usuários que serão aumejados e trabalhados, quando
falamos sobre gamificação.

Esta fase é um processo de definição de nicho sobre onde a gamificação vai atuar
, quanto a
usuários, dentro da Rede Social About? Quais serão os passos utilizados para queeste público
seja atingindo?

Alguns exemplos de tipos de usuário se encontram a seguir:

\begin{itemize}
    \item Companhias que desejam que seus trabalhadores atinjam determinadas métricas
        ao fim de cada mês;
    \item Educadores e políticos que querem utilizar conhecimento para criar impáctos
        sociais;
    \item Indivíduos que são apaixonados por gamificação, games e desenvolvimento próprio.
\end{itemize}
 
Desta maneira, será possível realizar um projeto de gamificação focado ao definir o tipo
de usuários. Pois, a partir daí, será possível identificar quais caminhos são mais vantajasos
quanto a escolha das motivações básicas que serão utilizadas ao longo das quatro fases.

\subsection{Define Desired Actions}
\label{sub:define_desired_actions}
A definição das ações desejadas são todas as iniciativas tomadas pelo usuário que o levam a caminhar para
o Win Stade(Estado de Vitória), seja ela em qual fase for. Sendo assim, a Rede Social
About terá alguns pontos que serão definidos como os desejados. Estes serão desenhados
até que o estado de vitória seja definido. Assim, para as quatro fases serão definidas
ações diferentes. Alguns exemplos de ações que podem ser escolhidas serão apresentadas
a seguir.

Ações na fase da descoberta:
\begin{itemize}
    \item Conhecer a Rede Social About;
    \item Clicar no link da Rede Social About
    \item Conhecer as features oferecidas pela Rede Social
\end{itemize}


Ações na fase de reconhecimento do projeto: 
\begin{itemize}
    \item Executar o tutorial de uso da About;
    \item Compartilhar a Rede Social About com os amigos
    \item Adicionar foto e email na network
    \item Permitir a escrição na lista de email
\end{itemize}

Já para a fase de construção do projeto, os seguintes pontos podem ser um
exêmplo:

\begin{itemize}
    \item Fazer login diariamente na network
    \item Abrir semanalmente os emails enviados pela network
    \item Compartilhar abouts com os amigos
    \item Participar de grupos no facebook sobre a rede social about
    \item Adquirir a versão prêmio da rede social about
    \item Inscrever em grupos de discussão sobre a rede social about
    \item Escrever mais de um about diariamente
    \item Votar em mais de vinte abouts diarios
\end{itemize}

Por fim, na fase de fim de jogo, alguns exemplos de construção podem ser dados. 
Eles são os seguintes:
\begin{itemize}
    \item Se tornar contribuidor da Rede Social About;
    \item Fazer parte da equipe de desenvolvedores da About
    \item Propor melhorias para a about
    \item Tornar-se moderador dos abouts
\end{itemize}

Estes exemplos ajudam e exclarecer como os objetivos podem ser alcançados. Elas 
definem um nível de
granularidade maior.

\subsection{Define Feedback Mechanics}
\label{sub:define_feedback_mechanics}
A definição de mecanismos de feedback são extremamente importantes para a exeperiência do usuário
com a network. Este é responsável por ilustrar e deixar bem claro para o usuário
, como ele está
prosseguindo no desenvolvimento do projeto.

Atualmente os usuários tem requirido feedbacks constantes, em tempo real, para as suas ações
realizadas. Sendo assim, é necessário que existam esses gatilhos em vários ponto
s da
Rede Social About e que o usuário possa entender rapidamente.

A seguir estão alguns exêmplos de como podem ser exclarecidos esses feedbacks pa ra o usuário:

\begin{itemize}
    \item Countdown Timers
    \item Desbloquear conteúdo da página
    \item Status de progresso na sidebar
    \item Verificação de qual era a melhor escolha
    \item Vídeo embutido
    \item Barra de pontos de status
    \item Certificados
    \item Medalhas
    \item Gráficos de desempenho
\end{itemize}

Assim, com exemplos dessa maneira, é possível que o usuário verifique o quanto suas atividades estão
sendo aproveitadas.

\section{Incentives And Rewards}
\label{sub:incentives_and_rewards}
O sistema de incentivos e recompensas fecham o ciclo do dash board, que fazem com que 
os usuários se sitam motivados a alcançar cada estado de vitória. Eles ajudam a 
indicar
o quanto ainda falta para que o estado seja aumejado.

\begin{itemize}
    \item Status Points
    \item Símbolos de vitórias
    \item Conhecer os desenvolvedores da about
    \item Ter acesso a arquivos confidenciais
    \item Descontos nos produtos
\end{itemize}

\section{Objetos de Gamificação}
\label{sec:objetodegamificacao}
Os objetos de gamificação serão os pontos da rede social em será aplicado o framework,
com os objetivos de atingir alguma meta de negócio.

Os objetivos de gamificação são os seguintes:

\begin{itemize}
    \item Fazer com que o usuário escreva mais abouts;
    \item Fazer com que o usuário julgue mais abouts;
    \item Fazer com que o usuário convide amigos que não estão cadastrados na about;
\end{itemize}

\section{Processo de Desenvolvimento de Software}
\label{sec:processo_de_desenvolvimento_de_software}
Este capítulo apresenta conceitos sobre processos de produção de software e alguns modelos de processo genéricos, como cascata, evolucionário e outros.

Pressman (2011) traz uma boa definição para processos software, afirmando que trata-se é um conjunto de metodologias definidas das atividades, ações e tarefas definidas para o desenvolvimento de um software de alta qualidade.

A Engenharia de Software (ESW) está diretamente ligada com o processo. Seu compromisso é garantir que haja qualidade de produção, pois, qualidade do processo resulta em qualidade no produto. 

De acordo com Pressman (2011), os engenheiros de software tem que possuir criatividade e conhecimento para que sejam capazes de analisar a demanda do mercado e então alinhar o desenvolvimento da melhor maneira possível, para que o resultado seja um produto de qualidade e entregue no prazo de tempo esperado.

 Para detalhar um processo, Pressman (2011) diz que um procedimento de software consiste em definir qual metodologia será utilizada para executar as diversas ações que o compõem. Além disso, as ações são um conjunto de tarefas de trabalho a ser completadas, artefatos de software que serão produzidos, marcos utilizados para indicar estado do processo e fatores para garantir a qualidade. 

Para Sommerville (2007), existem muitas metodologias diferentes para se desenvolver um software, porém, algumas atividades são fundamentais para qualquer desenvolvimento, dentre elas são citadas as seguintes:

\begin{enumerate}
    \item Especificação de software: É preciso definir a funcionalidade do software e as restrições em sua operação;

    \item Projeto e implementação de software: Deve ser produzido o software de maneira que cumpra a especificação;

    \item Validade de software: O software precisa ser validado para garantir que ele faz o que o cliente quer que seja feito;

    \item Evolução de software: O software precisa evoluir para atender as necessidades mutáveis do cliente.
\end{enumerate}


Sommerville (2007) afirma que um modelo de software é uma representação de um processo de forma abstrata. Existem vários modelos genéricos de produção de software, e que podem ser adaptados. Porém alguns podem ser destacados:

\begin{itemize}
    \item Modelo em cascata: Este modelo considera as atividades de especificação, desenvolvimento, validação e evolução que são fundamentais para o processo, e as representa como fases separadas do processo, como a especificação dos requisitos, projeto do software, implementação, testes e assim por diante;

    \item Desenvolvimento evolucionário: Essa abordagem intercala as atividades de especificação, desenvolvimento e validação. Um sistema inicial é rapidamente desenvolvido a partir de especificações abstratas, que são então refinadas com informações do cliente, para produzir um sistema que satisfaça suas necessidades;

    \item Desenvolvimento formal de sistemas: Essa abordagem se baseia na produção de uma especificação formal de matemática do sistema na transformação dessa especificação, utilizando-se de métodos matemáticos para construir um programa. A verificação de componentes do sistema é realizada mediante argumentos matemáticos, mostrando que eles atendem as suas especificações;

    \item Desenvolvimento orientado a reuso: Essa abordagem tem como base a existência de um número significativo de componentes reutilizáveis. O processo de desenvolvimento de sistemas se concentra na integração desses componentes em um sistema, em vez de proceder o desenvolvimento a partir do zero.
\end{itemize}

Dentre esses quatro modelos, será tratado neste trabalho o modelo de desenvolvimento evolucionário. Pois as metodologias ágeis, metodologia foco deste estudo, se enquadra neste modelo



\section{Metodologia de Desenvolvimento Ágil}
\label{sec:section_name}

Nesse capítulo são apresentados conceitos sobre as metodologias de desenvolvimento ágil, bem como algumas filosofias abordadas por ela.
As metodologias de desenvolvimento ágeis podem ser expressadas por uma agregação de princípios e valores, onde buscam priorizar o seguinte conjunto de valores, de acordo com BECK (2001), descritos no Manifesto para o Desenvolvimento Ágil:


\begin{itemize}
    \item Indivíduos e interações mais que processos e ferramentas;
    \item Software em funcionamento mais que documentação abrangente;
    \item Colaboração com o cliente mais que negociação de contratos;
    \item Responder a mudanças mais que seguir um plano.
\end{itemize}

Para BECK (2001), por mais que exista valor nos itens à direita da frase, os itens a esquerda são mais valorizados nessa metodologia.

Segundo Pressman (2011), o surgimento das metodologias ágeis foi uma estratégia de sanar os empecilhos vigentes da engenharia de software tradicional. A metodologia apresenta inúmeros benefícios, porém, não pode ser aplicada em qualquer projeto de desenvolvimento, produtos pessoas e situações. Para cada ambiente existe uma metodologia que atenda melhor as suas necessidades.

Nessa linha de pensamento, Pressman (2011) diz que os métodos ágeis são muito convenientes para projetos que estão em constante mudança e que as necessidades são alteradas em um período de tempo curto. Isso é permitido devido à política de entrega de uma pequena parte do software pronta para o cliente, com ciclos curtos, chamados de interações.

Como consequência das filosofias ágeis, SOARES (pag. 5) observou que há uma grande preocupação com a otimização do tempo, ao ter um esforço menor com documentação desnecessária para alcançar o produto final e mais devoção para na produção.

De acordo com SOARES (pag. 5), o desenvolvimento ágil também é conhecido por ser adaptativo ao invés de preditivo. Nos processos tradicionais, o planejamento e escopo eram inteiramente definidos no começo do projeto. Esse planejamento tem um custo elevado, além de ser extremamente complexo. O seu escopo não pode ser alterado durante o processo, fato que pode trazer a insatisfação do cliente, pois as suas necessidades podem mudar com o passar do tempo. Se houver algum erro no planejamento, ele será encontrado apenas no entrega final para o cliente. Como já dito por Pressman (2011), o modelo ágil soluciona esses problemas com entregas de produtos funcionais para o cliente em ciclos curtos. A cada ciclo o cliente pode avaliar se o que foi desenvolvido está dentro das suas necessidades e quais as necessidades futuras são necessárias para a próxima entrega. Esta política diminui o risco da perca de trabalho por um planejamento errado, por um requisito mal interpretado ou mal implementado e por retrabalho ao implementar novamente o mesmo requerimento do cliente.

Essa metodologia vem sendo mais presente em projetos de desenvolvimento de software. Segundo DUBAKOV e STEVENS (2008), no próprio ano de 2008, 70\% das organizações estavam utilizando métodos ágeis para a produção de software.
